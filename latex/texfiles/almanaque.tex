\documentclass[11pt, a4paper, twoside]{article}
\usepackage[T1]{fontenc}
\usepackage[utf8]{inputenc}
\usepackage{amssymb,amsmath}
\usepackage[portuguese]{babel}
\usepackage{comment}
\usepackage{datetime}
\usepackage[pdfusetitle]{hyperref}
\usepackage[all]{xy}
\usepackage{graphicx}
\addtolength{\parskip}{.5\baselineskip}

%aqui comeca o que eu fiz de verdade, o resto veio e eu to com medo de tirar
\usepackage{xcolor}
\usepackage{listings} %biblioteca pro codigo
\usepackage{color}    %deixa o codigo colorido bonitinho
\usepackage[portrait, left=2cm, right=1.5cm, top=2cm, bottom=2cm]{geometry} %pra deixar a margem do jeito que o brasil gosta

\definecolor{gray}{rgb}{0.2, 0.2, 0.2} %cor pros comentarios
\renewcommand{\footnotesize}{\small} %isso eh pra mudar o tamanho da fonte do codigo
% \setlength{\columnseprule}{0.2pt} %barra separando as duas colunas
% \setlength{\columnsep}{20pt} %distancia do texto ate a barra

\lstset{ %opcoes pro codigo
breaklines=true,
keywordstyle=\color{blue}\bfseries,
commentstyle=\color{gray},
breakatwhitespace=true,
language=C++,
%frame=single, % nao sei se gosto disso ou nao
numbers=none,
rulecolor=\color{black},
showstringspaces=false
stringstyle=\color{purple},
tabsize=4,
basicstyle=\ttfamily\footnotesize, % fonte
}
\lstset{literate=
%   *{0}{{{\color{red!20!violet}0}}}1
%    {1}{{{\color{red!20!violet}1}}}1
%    {2}{{{\color{red!20!violet}2}}}1
%    {3}{{{\color{red!20!violet}3}}}1
%    {4}{{{\color{red!20!violet}4}}}1
%    {5}{{{\color{red!20!violet}5}}}1
%    {6}{{{\color{red!20!violet}6}}}1
%    {7}{{{\color{red!20!violet}7}}}1
%    {8}{{{\color{red!20!violet}8}}}1
%    {9}{{{\color{red!20!violet}9}}}1
%	 {l}{$\text{l}$}1
	{~}{$\sim$}{1} % ~ bonitinho
}

\title{NTJ \\ UDESC}
\author{Eric Grochowicz, Enzo de Almeida Rodrigues e João Marcos de Oliveira}


\begin{document}
% \twocolumn
\date{\today}
\maketitle


\renewcommand{\contentsname}{Índice} %troca o nome do indice para indice
\tableofcontents
\clearpage


%%%%%%%%%%%%%%%%%%%%
%
% Strings
%
%%%%%%%%%%%%%%%%%%%%

\section{Strings}

\subsection{Automato de Aho Corasick}
\begin{lstlisting}
// Fonte: https://github.com/shahjalalshohag/code-library
//
// Faz coisarada

 const int N = 3e5 + 5;
 
 struct AC {
     int N, P;
     const int A = 26;
     vector<vector<int>> next;
     vector<int> link, out_link;
     vector<vector<int>> out;
     AC() : N(0), P(0) { node(); }
     int node() {
         next.emplace_back(A, 0);
         link.emplace_back(0);
         out_link.emplace_back(0);
         out.emplace_back(0);
         return N++;
     }
     inline int get(char c) { return c - 'a'; }
     int add_pattern(const string T) {
         int u = 0;
         for (auto c : T) {
             if (!next[u][get(c)]) next[u][get(c)] = node();
             u = next[u][get(c)];
         }
         out[u].push_back(P);
         return P++;
     }
     void compute() {
         queue<int> q;
         for (q.push(0); !q.empty();) {
             int u = q.front();
             q.pop();
             for (int c = 0; c < A; ++c) {
                 int v = next[u][c];
                 if (!v) next[u][c] = next[link[u]][c];
                 else {
                     link[v] = u ? next[link[u]][c] : 0;
                     out_link[v] = out[link[v]].empty() ? out_link[link[v]] : link[v];
                     q.push(v);
                 }
             }
         }
     }
     int advance(int u, char c) {
         while (u && !next[u][get(c)]) u = link[u];
         u = next[u][get(c)];
         return u;
     }
 };
 
 /*
 int32_t main() {
     ios_base::sync_with_stdio(0);
     cin.tie(0);
     auto st = clock();
     int t, cs = 0;
     cin >> t;
     while (t--) {
         int n;
         cin >> n;
         vector<string> v;
         for (int i = 0; i < n; i++) {
             string s;
             cin >> s;
             v.push_back(s);
         }
         sort(v.begin(), v.end());
         v.erase(unique(v.begin(), v.end()), v.end());
         AC aho;
         vector<int> len(n + 3, 0);
         for (auto s : v) {
             len[aho.add_pattern(s)] = s.size();
         }
         aho.compute();
         string s;
         cin >> s;
         n = s.size();
         vector<int> dp(n, n + 10);
         int u = 0;
         for (int i = 0; i < n; i++) {
             char c = s[i];
             u = aho.advance(u, c);
             for (int v = u; v; v = aho.out_link[v]) {
                 for (auto p : aho.out[v]) {
                     dp[i] = min(dp[i], (i - len[p] >= 0 ? dp[i - len[p]] : 0) + 1);
                 }
             }
         }
         cout << "Case " << ++cs << ": ";
         if (dp[n - 1] == n + 10) {
             cout << "impossible\n";
         } else {
             cout << dp[n - 1] << '\n';
         }
     }
     cout << 1.0 * (clock() - st) / 1000 << '\n';
     return 0;
 }
 */
\end{lstlisting}

\subsection{Hashing estatico (sem update)}
\begin{lstlisting}
// Usa o mint
//
// Build: O(n)
// Query: O(1)

// para usar 1 mod apenas
// using Hash = mint;

 const int mod1 = 998244353;
 const int mod2 = 1e9 + 7;
 using mint1 = Mint<mod1>;
 using mint2 = Mint<mod2>;
 using Hash = pair<mint1, mint2>;
 
 Hash operator*(Hash a, Hash o) {
     return {a.first * o.first, a.second * o.second};
 }
 Hash operator+(Hash a, Hash o) {
     return {a.first + o.first, a.second + o.second};
 }
 Hash operator-(Hash a, Hash o) {
     return {a.first - o.first, a.second - o.second};
 }
 
 const int PRIME = 1000000 + (rng() % 1000000); // nao necessariamente primo
 
 const int maxn = 1e6 + 5;
 Hash P = {PRIME, PRIME};
 Hash invP = {mint1(1) / PRIME, mint2(1) / PRIME};
 Hash p[maxn], invp[maxn];
 
 void initPrime() {
     p[0] = invp[0] = Hash(1, 1);
     for (int i = 1; i < N; i++) {
         p[i] = p[i - 1] * P;
         invp[i] = invp[i - 1] * invP;
     }
 }
 
 template<typename obj> struct Hashing {
     int N;
     vector<Hash> hsh;
     Hashing () {}
     Hashing(obj s) : N(size(s)), hsh(N + 1) {
         for (int i = 0; i < N; i++) {
             hsh[i + 1] = hsh[i] + (p[i + 1] * Hash(s[i], s[i]));
         }
     }
     Hash operator()(int l, int r) const {
         l++; r++;
         return (hsh[r] - hsh[l - 1]) * invp[l - 1];
     }
 };
 
 template<typename obj> struct revHashing { // util pra verificar palindromos e afins
     int N;
     vector<Hash> hsh;
     revHashing () {}
     revHashing(obj s) : N(size(s)), hsh(N + 1) {
         for (int i = N - 1; i >= 0; i--) {
             hsh[i] = hsh[i + 1] + (p[N - i] * Hash(s[i], s[i]));
         }
     }
     Hash operator()(int l, int r) const {
         return (hsh[l] - hsh[r + 1]) * invp[N - r - 1];
     }
 };
\end{lstlisting}

\subsection{KMP}
\begin{lstlisting}
// Fonte: https://github.com/shahjalalshohag/code-library
//
// String matching

 const int N = 3e5 + 9;
 
 // returns the longest proper prefix array of pattern p
 // where lps[i]=longest proper prefix which is also suffix of p[0...i]
 vector<int> build_lps(string p) {
     int sz = p.size();
     vector<int> lps;
     lps.assign(sz + 1, 0);
     int j = 0;
     lps[0] = 0;
     for (int i = 1; i < sz; i++) {
         while (j >= 0 && p[i] != p[j]) {
             if (j >= 1)
                 j = lps[j - 1];
             else
                 j = -1;
         }
         j++;
         lps[i] = j;
     }
     return lps;
 }
 vector<int> ans;
 // returns matches in vector ans in 0-indexed
 void kmp(vector<int> lps, string s, string p) {
     int psz = p.size(), sz = s.size();
     int j = 0;
     for (int i = 0; i < sz; i++) {
         while (j >= 0 && p[j] != s[i])
             if (j >= 1)
                 j = lps[j - 1];
             else
                 j = -1;
         j++;
         if (j == psz) {
             j = lps[j - 1];
             // pattern found in string s at position i-psz+1
             ans.push_back(i - psz + 1);
         }
         // after each loop we have j=longest common suffix of s[0..i] which is
         // also prefix of p
     }
 }
 
 /*
 int main() {
     int i, j, k, n, m, t;
     cin >> t;
     while (t--) {
         string s, p;
         cin >> s >> p;
         vector<int> lps = build_lps(p);
         kmp(lps, s, p);
         if (ans.empty())
             cout << "Not Found\n";
         else {
             cout << ans.size() << endl;
             for (auto x : ans) cout << x << ' ';
             cout << endl;
         }
         ans.clear();
         cout << endl;
     }
     return 0;
 }
 */
\end{lstlisting}

\subsection{Suffix Automaton}
\begin{lstlisting}
// Fonte: https://github.com/shahjalalshohag/code-library
//
// Faz coisarada

 const int N = 3e5 + 9;
 
 // len -> largest string length of the corresponding endpos-equivalent class
 // link -> longest suffix that is another endpos-equivalent class.
 // firstpos -> 1 indexed end position of the first occurrence of the largest
 // string of that node minlen(v) -> smallest string of node v = len(link(v)) + 1
 // terminal nodes -> store the suffixes
 struct SuffixAutomaton {
     struct node {
         int len, link, firstpos;
         map<char, int> nxt;
     };
     int sz, last;
     vector<node> t;
     vector<int> terminal;
     vector<long long> dp;
     vector<vector<int>> g;
     SuffixAutomaton() {}
     SuffixAutomaton(int n) {
         t.resize(2 * n);
         terminal.resize(2 * n, 0);
         dp.resize(2 * n, -1);
         sz = 1;
         last = 0;
         g.resize(2 * n);
         t[0].len = 0;
         t[0].link = -1;
         t[0].firstpos = 0;
     }
     void extend(char c) {
         int p = last;
         if (t[p].nxt.count(c)) {
             int q = t[p].nxt[c];
             if (t[q].len == t[p].len + 1) {
                 last = q;
                 return;
             }
             int clone = sz++;
             t[clone] = t[q];
             t[clone].len = t[p].len + 1;
             t[q].link = clone;
             last = clone;
             while (p != -1 && t[p].nxt[c] == q) {
                 t[p].nxt[c] = clone;
                 p = t[p].link;
             }
             return;
         }
         int cur = sz++;
         t[cur].len = t[last].len + 1;
         t[cur].firstpos = t[cur].len;
         p = last;
         while (p != -1 && !t[p].nxt.count(c)) {
             t[p].nxt[c] = cur;
             p = t[p].link;
         }
         if (p == -1)
             t[cur].link = 0;
         else {
             int q = t[p].nxt[c];
             if (t[p].len + 1 == t[q].len)
                 t[cur].link = q;
             else {
                 int clone = sz++;
                 t[clone] = t[q];
                 t[clone].len = t[p].len + 1;
                 while (p != -1 && t[p].nxt[c] == q) {
                     t[p].nxt[c] = clone;
                     p = t[p].link;
                 }
                 t[q].link = t[cur].link = clone;
             }
         }
         last = cur;
     }
     void build_tree() {
         for (int i = 1; i < sz; i++) g[t[i].link].push_back(i);
     }
     void build(string &s) {
         for (auto x : s) {
             extend(x);
             terminal[last] = 1;
         }
         build_tree();
     }
     long long cnt(int i) { // number of times i-th node occurs in the string
         if (dp[i] != -1) return dp[i];
         long long ret = terminal[i];
         for (auto &x : g[i]) ret += cnt(x);
         return dp[i] = ret;
     }
 };
 
 /*
 int32_t main() {
     ios_base::sync_with_stdio(0);
     cin.tie(0);
     int t;
     cin >> t;
     while (t--) {
         string s;
         cin >> s;
         int n = s.size();
         SuffixAutomaton sa(n);
         sa.build(s);
         long long ans = 0; // number of unique substrings
         for (int i = 1; i < sa.sz; i++) ans += sa.t[i].len - sa.t[sa.t[i].link].len;
         cout << ans << '\n';
     }
     return 0;
 }
 */
\end{lstlisting}

\clearpage


%%%%%%%%%%%%%%%%%%%%
%
% Problemas
%
%%%%%%%%%%%%%%%%%%%%

\section{Problemas}

\subsection{Kth digito na string infinita de digitos}
\begin{lstlisting}
// Retorna qual o numero e qual o algarismo do Kth digito
// na string infinita dos numeros naturais (12345678910111213...)
// Complexidade: O(log_10(k))

 pair<ll, ll> kthdig(ll k) {
     ll qtd = 1, num_alg = 1, base = 1;
     while (1) {
         ll add = (9 * base) * num_alg;
         if (qtd + add < k) {
             qtd += add;
         } else
             break;
         base *= 10, num_alg++;
     }
     ll algarismo = (k - qtd) % num_alg;
     ll numero = (k - qtd) / num_alg + base;
     return {numero, algarismo};
 }
\end{lstlisting}

\clearpage


%%%%%%%%%%%%%%%%%%%%
%
% Estruturas
%
%%%%%%%%%%%%%%%%%%%%

\section{Estruturas}

\subsection{Fenwick Tree}
\begin{lstlisting}
// Processas queries de operacao com inverso
//
// Build: O(n)
// Query: O(log(n))
// Update: O(log(n))

 typedef long long ll;
 
 struct fenwick {
     vector<ll> bit;
     fenwick(int n) { bit.assign(n + 1, 0); }
     fenwick(vector<ll> &v) {
         int n = v.size();
         bit.assign(n + 1, 0);
         for (int i = 1; i <= n; i++) bit[i] = v[i - 1];
         for (int i = 1; i <= n; i++) {
             int j = i + (i & -i);
             if (j <= n) bit[j] += bit[i];
         }
     }
     ll query(int i) {
         ll res = 0;
         for (; i; i -= (i & -i)) res += bit[i];
         return res;
     }
     ll query(int l, int r) { return query(r) - query(l - 1); }
     void update(int i, ll d) {
         for (; i && i < (int)bit.size(); i += (i & -i)) bit[i] += d;
     }
 };
\end{lstlisting}

\subsection{Segment Tree Beats}
\begin{lstlisting}
// Faz coisarada

 const int MAXN = 200001; // 1-based
 
 int N;
 ll A[MAXN];
 
 struct Node {
     ll sum;  // Sum tag
     ll max1; // Max value
     ll max2; // Second Max value
     ll maxc; // Max value count
     ll min1; // Min value
     ll min2; // Second Min value
     ll minc; // Min value count
     ll lazy; // Lazy tag
 } T[MAXN * 4];
 
 void merge(int t) {
     // sum
     T[t].sum = T[t << 1].sum + T[t << 1 | 1].sum;
 
     // max
     if (T[t << 1].max1 == T[t << 1 | 1].max1) {
         T[t].max1 = T[t << 1].max1;
         T[t].max2 = max(T[t << 1].max2, T[t << 1 | 1].max2);
         T[t].maxc = T[t << 1].maxc + T[t << 1 | 1].maxc;
     } else {
         if (T[t << 1].max1 > T[t << 1 | 1].max1) {
             T[t].max1 = T[t << 1].max1;
             T[t].max2 = max(T[t << 1].max2, T[t << 1 | 1].max1);
             T[t].maxc = T[t << 1].maxc;
         } else {
             T[t].max1 = T[t << 1 | 1].max1;
             T[t].max2 = max(T[t << 1].max1, T[t << 1 | 1].max2);
             T[t].maxc = T[t << 1 | 1].maxc;
         }
     }
 
     // min
     if (T[t << 1].min1 == T[t << 1 | 1].min1) {
         T[t].min1 = T[t << 1].min1;
         T[t].min2 = min(T[t << 1].min2, T[t << 1 | 1].min2);
         T[t].minc = T[t << 1].minc + T[t << 1 | 1].minc;
     } else {
         if (T[t << 1].min1 < T[t << 1 | 1].min1) {
             T[t].min1 = T[t << 1].min1;
             T[t].min2 = min(T[t << 1].min2, T[t << 1 | 1].min1);
             T[t].minc = T[t << 1].minc;
         } else {
             T[t].min1 = T[t << 1 | 1].min1;
             T[t].min2 = min(T[t << 1].min1, T[t << 1 | 1].min2);
             T[t].minc = T[t << 1 | 1].minc;
         }
     }
 }
 
 void push_add(int t, int tl, int tr, ll v) {
     if (v == 0) {
         return;
     }
     T[t].sum += (tr - tl + 1) * v;
     T[t].max1 += v;
     if (T[t].max2 != -llINF) {
         T[t].max2 += v;
     }
     T[t].min1 += v;
     if (T[t].min2 != llINF) {
         T[t].min2 += v;
     }
     T[t].lazy += v;
 }
 
 // corresponds to a chmin update
 void push_max(int t, ll v, bool l) {
     if (v >= T[t].max1) {
         return;
     }
     T[t].sum -= T[t].max1 * T[t].maxc;
     T[t].max1 = v;
     T[t].sum += T[t].max1 * T[t].maxc;
     if (l) {
         T[t].min1 = T[t].max1;
     } else {
         if (v <= T[t].min1) {
             T[t].min1 = v;
         } else if (v < T[t].min2) {
             T[t].min2 = v;
         }
     }
 }
 
 // corresponds to a chmax update
 void push_min(int t, ll v, bool l) {
     if (v <= T[t].min1) {
         return;
     }
     T[t].sum -= T[t].min1 * T[t].minc;
     T[t].min1 = v;
     T[t].sum += T[t].min1 * T[t].minc;
     if (l) {
         T[t].max1 = T[t].min1;
     } else {
         if (v >= T[t].max1) {
             T[t].max1 = v;
         } else if (v > T[t].max2) {
             T[t].max2 = v;
         }
     }
 }
 
 void pushdown(int t, int tl, int tr) {
     if (tl == tr) return;
     // sum
     int tm = (tl + tr) >> 1;
     push_add(t << 1, tl, tm, T[t].lazy);
     push_add(t << 1 | 1, tm + 1, tr, T[t].lazy);
     T[t].lazy = 0;
 
     // max
     push_max(t << 1, T[t].max1, tl == tm);
     push_max(t << 1 | 1, T[t].max1, tm + 1 == tr);
 
     // min
     push_min(t << 1, T[t].min1, tl == tm);
     push_min(t << 1 | 1, T[t].min1, tm + 1 == tr);
 }
 
 void build(int t = 1, int tl = 0, int tr = N - 1) {
     T[t].lazy = 0;
     if (tl == tr) {
         T[t].sum = T[t].max1 = T[t].min1 = A[tl];
         T[t].maxc = T[t].minc = 1;
         T[t].max2 = -llINF;
         T[t].min2 = llINF;
         return;
     }
 
     int tm = (tl + tr) >> 1;
     build(t << 1, tl, tm);
     build(t << 1 | 1, tm + 1, tr);
     merge(t);
 }
 
 void update_add(int l, int r, ll v, int t = 1, int tl = 0, int tr = N - 1) {
     if (r < tl || tr < l) {
         return;
     }
     if (l <= tl && tr <= r) {
         push_add(t, tl, tr, v);
         return;
     }
     pushdown(t, tl, tr);
 
     int tm = (tl + tr) >> 1;
     update_add(l, r, v, t << 1, tl, tm);
     update_add(l, r, v, t << 1 | 1, tm + 1, tr);
     merge(t);
 }
 
 void update_chmin(int l, int r, ll v, int t = 1, int tl = 0, int tr = N - 1) {
     if (r < tl || tr < l || v >= T[t].max1) {
         return;
     }
     if (l <= tl && tr <= r && v > T[t].max2) {
         push_max(t, v, tl == tr);
         return;
     }
     pushdown(t, tl, tr);
 
     int tm = (tl + tr) >> 1;
     update_chmin(l, r, v, t << 1, tl, tm);
     update_chmin(l, r, v, t << 1 | 1, tm + 1, tr);
     merge(t);
 }
 
 void update_chmax(int l, int r, ll v, int t = 1, int tl = 0, int tr = N - 1) {
     if (r < tl || tr < l || v <= T[t].min1) {
         return;
     }
     if (l <= tl && tr <= r && v < T[t].min2) {
         push_min(t, v, tl == tr);
         return;
     }
     pushdown(t, tl, tr);
 
     int tm = (tl + tr) >> 1;
     update_chmax(l, r, v, t << 1, tl, tm);
     update_chmax(l, r, v, t << 1 | 1, tm + 1, tr);
     merge(t);
 }
 
 ll query_sum(int l, int r, int t = 1, int tl = 0, int tr = N - 1) {
     if (r < tl || tr < l) {
         return 0;
     }
     if (l <= tl && tr <= r) {
         return T[t].sum;
     }
     pushdown(t, tl, tr);
 
     int tm = (tl + tr) >> 1;
     return query_sum(l, r, t << 1, tl, tm) + query_sum(l, r, t << 1 | 1, tm + 1, tr);
 }
 
 /*
 int main() {
     int Q;
 
     cin >> N >> Q;
     for (int i = 0; i < N; i++) {
         cin >> A[i];
     }
     build();
     for (int q = 0; q < Q; q++) {
         int t;
         cin >> t;
         if (t == 0) {
             int l, r;
             ll x;
             cin >> l >> r >> x;
             update_chmin(l, r - 1, x);
         } else if (t == 1) {
             int l, r;
             ll x;
             cin >> l >> r >> x;
             update_chmax(l, r - 1, x);
         } else if (t == 2) {
             int l, r;
             ll x;
             cin >> l >> r >> x;
             update_add(l, r - 1, x);
         } else if (t == 3) {
             int l, r;
             cin >> l >> r;
             cout << query_sum(l, r - 1) << '\n';
         }
     }
 }
 */
\end{lstlisting}

\clearpage


%%%%%%%%%%%%%%%%%%%%
%
% Grafos
%
%%%%%%%%%%%%%%%%%%%%

\section{Grafos}

\subsection{Binary Lifting}
\begin{lstlisting}
// Binary Lifting pra LCA
//
// Computa Lowest Common Ancestor e faz queries de k-esimo ancestral
//
// Build(): O(n log(n))
// Lca(): O(log(n))
// Kth(): O(log(n))
//
// up[u][i] = (2 ^ i)-esimo pai do u

 struct BinaryLifting {
     vector<vector<int>> adj, up;
     vector<int> tin, tout;
     int N, LG, t;
 
     void dfs(int u, int p = -1) {
         tin[u] = t++;
         for (int i = 0; i < LG - 1; i++) up[u][i + 1] = up[up[u][i]][i];
         for (int v : adj[u]) if (v != p) {
             up[v][0] = u;
             dfs(v, u);
         }
         tout[u] = t++;
     }
 
     void build(int root, vector<vector<int>> adj2) {
         t = 1;
         N = size(adj2);
         LG = 32 - __builtin_clz(N);
         adj = adj2;
         tin = tout = vector<int>(N);
         up = vector (N, vector<int>(LG));
         up[root][0] = root;
         dfs(root);
     }
 
     bool ancestor(int u, int v) { return tin[u] <= tin[v] && tout[u] >= tout[v]; }
 
     int lca(int u, int v) {
         if (ancestor(u, v)) return u;
         if (ancestor(v, u)) return v;
         for (int i = LG - 1; i >= 0; i--) {
             if (!ancestor(up[u][i], v)) u = up[u][i];
         }
         return up[u][0];
     }
 
     int kth(int u, int k) {
         for (int i = 0; i < LG; i++) {
             if (k & (1 << i)) u = up[u][i];
         }
         return u;
     }
 
 } bl;
\end{lstlisting}

\subsection{Binary Lifting Query (em arestas)}
\begin{lstlisting}
// Resolve queries em arvore quando os valores
// estao nas arestas
//
// Build(): O(n log(n))
// query(): O(log(n))
//
// up[u][i] = (2 ^ i)-esimo pai do u
// st[u][i] = query ate (2 ^ i)-esimo pai do u

 struct BinaryLifting {
     vector<vector<ii>> adj;
     vector<vector<int>> up, st;
     vector<int> tin, tout;
     int N, LG, t;
 
     const int neutral = 0;
     int merge(int l, int r) { return l + r; }
 
     void dfs(int u, int p = -1) {
         tin[u] = t++;
         for (int i = 0; i < LG - 1; i++) {
             up[u][i + 1] = up[up[u][i]][i];
             st[u][i + 1] = merge(st[u][i], st[up[u][i]][i]);
         }
         for (auto [w, v] : adj[u])
             if (v != p) {
                 up[v][0] = u, st[v][0] = w;
                 dfs(v, u);
             }
         tout[u] = t++;
     }
 
     void build(int root, vector<vector<ii>> adj2) {
         t = 1;
         N = size(adj2);
         LG = 32 - __builtin_clz(N);
         adj = adj2;
         tin = tout = vector<int>(N);
         up = st = vector (N, vector<int>(LG, neutral));
         up[root][0] = root;
         dfs(root);
     }
 
     bool ancestor(int u, int v) { return tin[u] <= tin[v] && tout[u] >= tout[v]; }
 
     int query2(int u, int v) {
         if (ancestor(u, v)) return neutral;
         int ans = neutral;
         for (int i = LG - 1; i >= 0; i--) {
             if (!ancestor(up[u][i], v)) {
                 ans = merge(ans, st[u][i]);
                 u = up[u][i];
             }
         }
         return merge(ans, st[u][0]);
     }
 
     int query(int u, int v) {
         if (u == v) return neutral;
 #warning TRATAR ESSE CASO ACIMA
         return merge(query2(u, v), query2(v, u));
     }
 } bl;
 
\end{lstlisting}

\subsection{Binary Lifting Query (em nodos)}
\begin{lstlisting}
// Computa LCA e tambem resolve queries de operacoes
// associativas e comutativas em caminhos.
//
// Build(): O(n log(n))
// Query(): O(log(n))
// Lca(): O(log(n))
// Kth(): O(log(n))
//
// up[u][i] = (2 ^ i)-esimo pai do u
// st[u][i] = query ate (2 ^ i)-esimo pai do u (NAO INCLUI O U)

 struct BinaryLifting {
     vector<vector<int>> adj, up, st;
     vector<int> val, tin, tout;
     int N, LG, t;
 
     const int neutral = 0;
     int merge(int l, int r) { return l + r; }
 
     void dfs(int u, int p = -1) {
         tin[u] = t++;
         for (int i = 0; i < LG - 1; i++) {
             up[u][i + 1] = up[up[u][i]][i];
             st[u][i + 1] = merge(st[u][i], st[up[u][i]][i]);
         }
         for (int v : adj[u]) if (v != p) {
             up[v][0] = u, st[v][0] = val[u];
             dfs(v, u);
         }
         tout[u] = t++;
     }
 
     void build(int root, vector<vector<int>> adj2, vector<int> v) {
         t = 1;
         N = size(adj2);
         LG = 32 - __builtin_clz(N);
         adj = adj2;
         val = v;
         tin = tout = vector<int>(N);
         up = st = vector (N, vector<int>(LG, neutral));
         up[root][0] = root;
         st[root][0] = val[root];
         dfs(root);
     }
 
     bool ancestor(int u, int v) { return tin[u] <= tin[v] && tout[u] >= tout[v]; }
 
     int query2(int u, int v, bool include_lca) {
         if (ancestor(u, v)) return include_lca ? val[u] : neutral;
         int ans = val[u];
         for (int i = LG - 1; i >= 0; i--) {
             if (!ancestor(up[u][i], v)) {
                 ans = merge(ans, st[u][i]);
                 u = up[u][i];
             }
         }
         return include_lca ? merge(ans, st[u][0]) : ans;
     }
 
     int query(int u, int v) {
         if (u == v) return val[u];
         return merge(query2(u, v, 1), query2(v, u, 0));
     }
 
     int lca(int u, int v) {
         if (ancestor(u, v)) return u;
         if (ancestor(v, u)) return v;
         for (int i = LG - 1; i >= 0; i--) {
             if (!ancestor(up[u][i], v)) u = up[u][i];
         }
         return up[u][0];
     }
 
     int kth(int u, int k) {
         for (int i = 0; i < LG; i++) {
             if (k & (1 << i)) u = up[u][i];
         }
         return u;
     }
 
 } bl;
\end{lstlisting}

\subsection{Binary Lifting Query 2 (em nodos)}
\begin{lstlisting}
// Esse resolve queries de operacoes nao comutativas
// Levemente diferente do padrao
//
// Esse aqui resolve query de Kadani em arvore
// https://codeforces.com/contest/1843/problem/F2

 struct node {
     int pref, suff, sum, best;
     node() : pref(0), suff(0), sum(0), best(0) {}
     node(int x) : pref(x), suff(x), sum(x), best(x) {}
     node(int a, int b, int c, int d) : pref(a), suff(b), sum(c), best(d) {}
 };
 
 node merge(node &l, node &r) {
     int pref = max(l.pref, l.sum + r.pref);
     int suff = max(r.suff, r.sum + l.suff);
     int sum = l.sum + r.sum;
     int best = max(l.suff + r.pref, max(l.best, r.best));
     return node(pref, suff, sum, best);
 }
 
 struct BinaryLifting {
     vector<vector<int>> adj, up;
     vector<int> val, tin, tout;
     vector<vector<node>> st, st2;
     int N, LG, t;
 
     void build(int u, int p = -1) {
         tin[u] = t++;
         for (int i = 0; i < LG - 1; i++) {
             up[u][i + 1] = up[up[u][i]][i];
             st[u][i + 1] = merge(st[u][i], st[up[u][i]][i]);
             st2[u][i + 1] = merge(st2[up[u][i]][i], st2[u][i]);
         }
         for (int v : adj[u])
             if (v != p) {
                 up[v][0] = u;
                 st[v][0] = node(val[u]);
                 st2[v][0] = node(val[u]);
                 build(v, u);
             }
         tout[u] = t++;
     }
 
     void build(int root, vector<vector<int>> adj2, vector<int> v) {
         t = 1;
         N = size(adj2);
         LG = 32 - __builtin_clz(N);
         adj = adj2;
         val = v;
         tin = tout = vector<int>(N);
         up = vector(N, vector<int>(LG));
         st = st2 = vector(N, vector<node>(LG));
         up[root][0] = root;
         st[root][0] = node(val[root]);
         st2[root][0] = node(val[root]);
         build(root);
     }
 
     bool ancestor(int u, int v) { return tin[u] <= tin[v] && tout[u] >= tout[v]; }
 
     node query2(int u, int v, bool include_lca, bool invert) {
         if (ancestor(u, v)) return include_lca ? node(val[u]) : node();
         node ans = node(val[u]);
         for (int i = LG - 1; i >= 0; i--) {
             if (!ancestor(up[u][i], v)) {
                 if (invert)
                     ans = merge(st2[u][i], ans);
                 else
                     ans = merge(ans, st[u][i]);
                 u = up[u][i];
             }
         }
         if (!include_lca) return ans;
         return merge(ans, st[u][0]);
     }
 
     node query(int u, int v) {
         if (u == v) return node(val[u]);
         node l = query2(u, v, 1, 0);
         node r = query2(v, u, 0, 1);
         return merge(l, r);
     }
 
     int lca(int u, int v) {
         if (ancestor(u, v)) return u;
         if (ancestor(v, u)) return v;
         for (int i = LG - 1; i >= 0; i--) {
             if (!ancestor(up[u][i], v)) {
                 u = up[u][i];
             }
         }
         return up[u][0];
     }
 
 } bl, bl2;
\end{lstlisting}

\subsection{Bridges e Edge Biconnected Components}
\begin{lstlisting}
// Acha todas as pontes em O(n)
// Tambem constroi a arvore condensada, mantendo
// so as pontes como arestas e o resto comprimindo
// em nodos
//
// Salva no vetor bridges os pares {u, v} cujas arestas sao pontes

 typedef pair<int, int> ii;
 const int maxn = 2e5 + 5;
 int n, m;
 bool vis[maxn];
 int dp[maxn], dep[maxn];
 vector<int> adj[maxn];
 vector<ii> bridges;
 
 void dfs_dp(int u, int p = -1, int d = 0) {
     dp[u] = 0, dep[u] = d, vis[u] = 1;
     for (auto v : adj[u]) {
         if (v != p) {
             if (vis[v]) {
                 if (dep[v] < dep[u]) dp[v]--, dp[u]++;
             } else {
                 dfs_dp(v, u, d + 1);
                 dp[u] += dp[v];
             }
         }
     }
     if (dp[u] == 0 && p != -1) { // edge {u, p} eh uma ponte
         bridges.emplace_back(u, p);
     }
 }
 
 void find_bridges() {
     memset(vis, 0, n);
     for (int i = 0; i < n; i++) {
         if (!vis[i]) {
             dfs_dp(i);
         }
     }
 }
 
 // Edge Biconnected Components (requer todo codigo acima)
 
 int ebcc[maxn], ncc = 0;
 vector<int> adjbcc[maxn];
 
 void dfs_ebcc(int u, int p, int cc) {
     vis[u] = 1;
     if (dp[u] == 0 && p != -1) {
         cc = ++ncc;
     }
     ebcc[u] = cc;
     for (auto v : adj[u]) {
         if (!vis[v]) {
             dfs_ebcc(v, u, cc);
         }
     }
 }
 
 void build_ebcc_graph() {
     find_bridges();
     memset(vis, 0, n);
     for (int i = 0; i < n; i++) {
         if (!vis[i]) {
             dfs_ebcc(i, -1, ncc);
             ++ncc;
         }
     }
     // Opcao 1 - constroi o grafo condensado passando por todas as edges
     for (int u = 0; u < n; u++) {
         for (auto v : adj[u]) {
             if (ebcc[u] != ebcc[v]) {
                 adjbcc[ebcc[u]].emplace_back(ebcc[v]);
             } else {
                 // faz algo
             }
         }
     }
     // Opcao 2 - constroi o grafo condensado passando so pelas pontes
     for (auto [u, v] : bridges) {
         adjbcc[ebcc[u]].emplace_back(ebcc[v]);
         adjbcc[ebcc[v]].emplace_back(ebcc[u]);
     }
 }
\end{lstlisting}

\subsection{Dinic}
\begin{lstlisting}
// Fonte: https://github.com/shahjalalshohag/code-library
//
// Max Flow em O(V^3) ou O(E * sqrt(V)) em bipartido

 const int N = 5010;
 
 const long long inf = 1LL << 61;
 struct Dinic {
     struct edge {
         int to, rev;
         long long flow, w;
         int id;
     };
     int n, s, t, mxid;
     vector<int> d, flow_through;
     vector<int> done;
     vector<vector<edge>> g;
     Dinic() {}
     Dinic(int _n) {
         n = _n + 10;
         mxid = 0;
         g.resize(n);
     }
     void add_edge(int u, int v, long long w, int id = -1) {
         edge a = {v, (int)g[v].size(), 0, w, id};
         edge b = {u, (int)g[u].size(), 0, 0, -2}; // for bidirectional edges cap(b) = w
         g[u].emplace_back(a);
         g[v].emplace_back(b);
         mxid = max(mxid, id);
     }
     bool bfs() {
         d.assign(n, -1);
         d[s] = 0;
         queue<int> q;
         q.push(s);
         while (!q.empty()) {
             int u = q.front();
             q.pop();
             for (auto &e : g[u]) {
                 int v = e.to;
                 if (d[v] == -1 && e.flow < e.w) d[v] = d[u] + 1, q.push(v);
             }
         }
         return d[t] != -1;
     }
     long long dfs(int u, long long flow) {
         if (u == t) return flow;
         for (int &i = done[u]; i < (int)g[u].size(); i++) {
             edge &e = g[u][i];
             if (e.w <= e.flow) continue;
             int v = e.to;
             if (d[v] == d[u] + 1) {
                 long long nw = dfs(v, min(flow, e.w - e.flow));
                 if (nw > 0) {
                     e.flow += nw;
                     g[v][e.rev].flow -= nw;
                     return nw;
                 }
             }
         }
         return 0;
     }
     long long max_flow(int _s, int _t) {
         s = _s;
         t = _t;
         long long flow = 0;
         while (bfs()) {
             done.assign(n, 0);
             while (long long nw = dfs(s, inf)) flow += nw;
         }
         flow_through.assign(mxid + 10, 0);
         for (int i = 0; i < n; i++)
             for (auto e : g[i])
                 if (e.id >= 0) flow_through[e.id] = e.flow;
         return flow;
     }
 };
 
 /*
 int main() {
     int n, m;
     cin >> n >> m;
     Dinic F(n + 1);
     for (int i = 1; i <= m; i++) {
         int u, v, w;
         cin >> u >> v >> w;
         F.add_edge(u, v, w);
     }
     cout << F.max_flow(1, n) << '\n';
     return 0;
 }
 */
\end{lstlisting}

\subsection{Pontos de articulacao}
\begin{lstlisting}
// Fonte: https://github.com/shahjalalshohag/code-library
//
// O equivalente a pontes, em vertices
//
// Complexidade: O(n)

 const int N = 3e5 + 9;
 
 int T, low[N], dis[N], art[N];
 vector<int> g[N];
 void dfs(int u, int pre = 0) {
     low[u] = dis[u] = ++T;
     int child = 0;
     for (auto v : g[u]) {
         if (!dis[v]) {
             dfs(v, u);
             low[u] = min(low[u], low[v]);
             if (low[v] >= dis[u] && pre != 0) art[u] = 1;
             ++child;
         } else if (v != pre)
             low[u] = min(low[u], dis[v]);
     }
     if (pre == 0 && child > 1) art[u] = 1;
 }
 
 /*
 int32_t main() {
     ios_base::sync_with_stdio(0);
     cin.tie(0);
     while (1) {
         int n, m;
         cin >> n >> m;
         if (!n) break;
         while (m--) {
             int u, v;
             cin >> u >> v;
             g[u].push_back(v);
             g[v].push_back(u);
         }
         dfs(1);
         int ans = 0;
         for (int i = 1; i <= n; i++) ans += art[i];
         cout << ans << '\n';
         T = 0;
         for (int i = 1; i <= n; i++) low[i] = dis[i] = art[i] = 0, g[i].clear();
     }
     return 0;
 }
 */
\end{lstlisting}

\clearpage


%%%%%%%%%%%%%%%%%%%%
%
% Matematica
%
%%%%%%%%%%%%%%%%%%%%

\section{Matematica}

\subsection{Crivo de Eratostenes}
\begin{lstlisting}
// Computa numeros primos entre [2, n] em O(n)
//
// Crivo linear computando spf (smallest prime factor) pra cada numero
// x entre [2, n] e phi(x) (funcao totiente)
// Complexidade: O(n)

 int spf[maxn], phi[maxn];
 vector<int> primes;
 void sieve(int n) {
     phi[1] = 1;
     for (int i = 2; i <= n; i++) {
         if (spf[i] == 0) {
             spf[i] = i;
             primes.emplace_back(i);
             phi[i] = i - 1;
         }
         for (int j = 0; j < (int)primes.size() && i * primes[j] <= n && primes[j] <= spf[i]; j++) {
             spf[i * primes[j]] = primes[j];
             if (primes[j] < spf[i])
                 phi[i * primes[j]] = phi[i] * phi[primes[j]];
             else
                 phi[i * primes[j]] = phi[i] * primes[j];
         }
     }
 }
\end{lstlisting}

\subsection{Fast Fourier Transform}
\begin{lstlisting}
// Fonte: https://github.com/ShahjalalShohag/code-library
//
// Faz convolucao de dois polinomios
// Complexidade: O(n log(n))
//
// Testado e sem erro de precisao para MAXN = 3e5 e A_i = 1e9

 const int N = 3e5 + 9;
 
 const double PI = acos(-1);
 struct base {
     double a, b;
     base(double a = 0, double b = 0) : a(a), b(b) {}
     const base operator+(const base &c) const { return base(a + c.a, b + c.b); }
     const base operator-(const base &c) const { return base(a - c.a, b - c.b); }
     const base operator*(const base &c) const { return base(a * c.a - b * c.b, a * c.b + b * c.a); }
 };
 void fft(vector<base> &p, bool inv = 0) {
     int n = p.size(), i = 0;
     for (int j = 1; j < n - 1; ++j) {
         for (int k = n >> 1; k > (i ^= k); k >>= 1)
             ;
         if (j < i) swap(p[i], p[j]);
     }
     for (int l = 1, m; (m = l << 1) <= n; l <<= 1) {
         double ang = 2 * PI / m;
         base wn = base(cos(ang), (inv ? 1. : -1.) * sin(ang)), w;
         for (int i = 0, j, k; i < n; i += m) {
             for (w = base(1, 0), j = i, k = i + l; j < k; ++j, w = w * wn) {
                 base t = w * p[j + l];
                 p[j + l] = p[j] - t;
                 p[j] = p[j] + t;
             }
         }
     }
     if (inv)
         for (int i = 0; i < n; ++i) p[i].a /= n, p[i].b /= n;
 }
 vector<long long> multiply(vector<int> &a, vector<int> &b) {
     int n = a.size(), m = b.size(), t = n + m - 1, sz = 1;
     while (sz < t) sz <<= 1;
     vector<base> x(sz), y(sz), z(sz);
     for (int i = 0; i < sz; ++i) {
         x[i] = i < (int)a.size() ? base(a[i], 0) : base(0, 0);
         y[i] = i < (int)b.size() ? base(b[i], 0) : base(0, 0);
     }
     fft(x), fft(y);
     for (int i = 0; i < sz; ++i) z[i] = x[i] * y[i];
     fft(z, 1);
     vector<long long> ret(sz);
     for (int i = 0; i < sz; ++i) ret[i] = (long long)round(z[i].a);
     while ((int)ret.size() > 1 && ret.back() == 0) ret.pop_back();
     return ret;
 }
 
 /*
 long long ans[N];
 int32_t main() {
     ios_base::sync_with_stdio(0);
     cin.tie(0);
     int n, x;
     cin >> n >> x;
     vector<int> a(n + 1, 0), b(n + 1, 0), c(n + 1, 0);
     int nw = 0;
     a[0]++;
     b[n]++;
     long long z = 0;
     for (int i = 1; i <= n; i++) {
         int k;
         cin >> k;
         nw += k < x;
         a[nw]++;
         b[-nw + n]++;
         z += c[nw] + !nw;
         c[nw]++;
     }
     auto res = multiply(a, b);
     for (int i = n + 1; i < res.size(); i++) {
         ans[i - n] += res[i];
     }
     ans[0] = z;
     for (int i = 0; i <= n; i++) cout << ans[i] << ' ';
     cout << '\n';
     return 0;
 }
 */
\end{lstlisting}

\subsection{Pollard Rho}
\begin{lstlisting}
// Fonte: https://github.com/shahjalalshohag/code-library
//
// Fatora numeros ate 8*10^18
// Complexidade: O(n ^ (1/4))

 namespace PollardRho {
     mt19937 rnd(chrono::steady_clock::now().time_since_epoch().count());
     const int P = 1e6 + 9;
     ll seq[P];
     int primes[P], spf[P];
     inline ll add_mod(ll x, ll y, ll m) { return (x += y) < m ? x : x - m; }
     inline ll mul_mod(ll x, ll y, ll m) {
         ll res = __int128(x) * y % m;
         return res;
         // ll res = x * y - (ll)((long double)x * y / m + 0.5) * m;
         // return res < 0 ? res + m : res;
     }
     inline ll pow_mod(ll x, ll n, ll m) {
         ll res = 1 % m;
         for (; n; n >>= 1) {
             if (n & 1) res = mul_mod(res, x, m);
             x = mul_mod(x, x, m);
         }
         return res;
     }
     // O(it * (logn)^3), it = number of rounds performed
     inline bool miller_rabin(ll n) {
         if (n <= 2 || ((n & 1) ^ 1)) return (n == 2);
         if (n < P) return spf[n] == n;
         ll c, d, s = 0, r = n - 1;
         for (; !(r & 1); r >>= 1, s++) {
         }
         // each iteration is a round
         for (int i = 0; primes[i] < n && primes[i] < 32; i++) {
             c = pow_mod(primes[i], r, n);
             for (int j = 0; j < s; j++) {
                 d = mul_mod(c, c, n);
                 if (d == 1 && c != 1 && c != (n - 1)) return false;
                 c = d;
             }
             if (c != 1) return false;
         }
         return true;
     }
     void init() {
         int cnt = 0;
         for (int i = 2; i < P; i++) {
             if (!spf[i]) primes[cnt++] = spf[i] = i;
             for (int j = 0, k; (k = i * primes[j]) < P; j++) {
                 spf[k] = primes[j];
                 if (spf[i] == spf[k]) break;
             }
         }
     }
     // returns O(n^(1/4))
     ll pollard_rho(ll n) {
         while (1) {
             ll x = rnd() % n, y = x, c = rnd() % n, u = 1, v, t = 0;
             ll *px = seq, *py = seq;
             while (1) {
                 *py++ = y = add_mod(mul_mod(y, y, n), c, n);
                 *py++ = y = add_mod(mul_mod(y, y, n), c, n);
                 if ((x = *px++) == y) break;
                 v = u;
                 u = mul_mod(u, abs(y - x), n);
                 if (!u) return gcd(v, n);
                 if (++t == 32) {
                     t = 0;
                     if ((u = gcd(u, n)) > 1 && u < n) return u;
                 }
             }
             if (t && (u = gcd(u, n)) > 1 && u < n) return u;
         }
     }
 
     vector<ll> factorize(ll n) {
         if (n == 1) return vector<ll>();
         if (miller_rabin(n)) return vector<ll>{n};
         vector<ll> v, w;
         while (n > 1 && n < P) {
             v.push_back(spf[n]);
             n /= spf[n];
         }
         if (n >= P) {
             ll x = pollard_rho(n);
             v = factorize(x);
             w = factorize(n / x);
             v.insert(v.end(), w.begin(), w.end());
         }
         return v;
     }
 } // namespace PollardRho
 
 /*
 int32_t main() {
     ios_base::sync_with_stdio(0);
     cin.tie(0);
     PollardRho::init();
     int t;
     cin >> t;
     while (t--) {
         ll n;
         cin >> n;
         auto f = PollardRho::factorize(n);
         sort(f.begin(), f.end());
         cout << f.size() << ' ';
         for (auto x : f) cout << x << ' ';
         cout << '\n';
     }
     return 0;
 }
 */
\end{lstlisting}

\clearpage


%%%%%%%%%%%%%%%%%%%%
%
% Geometria
%
%%%%%%%%%%%%%%%%%%%%

\section{Geometria}

\subsection{Geometria inteiro}
\begin{lstlisting}
// Tudo que temos de geometria pra pontos inteiros

// Ponto com coordenadas inteiras e alguns metodos

 struct pt {
     ll x, y;
     pt() : x(0), y(0) {}
     pt(ll _x, ll _y) : x(_x), y(_y) {}
 
     pt operator*(const ll &b) { return pt(b * x, b * y); }
     pt operator-(const pt &b) { return pt(x - b.x, y - b.y); }
     pt operator+(const pt &b) { return pt(x + b.x, y + b.y); }
     ll operator*(const pt &b) { return x * b.x + y * b.y; }
     ll operator^(const pt &b) { return x * b.y - y * b.x; }
 
     bool operator<(const pt &p) const {
         if (x == p.x) return y < p.y;
         return x < p.x;
     }
     ll dist2(const pt &p) {
         ll dx = x - p.x;
         ll dy = y - p.y;
         return dx * dx + dy * dy;
     }
 
     friend ostream &operator<<(ostream &out, const pt &a) { return out << "(" << a.x << "," << a.y << ")"; }
     friend istream &operator>>(istream &in, pt &a) { return in >> a.x >> a.y; }
 };
 
 // Convex Hull
 // Algoritmo Graham's Scan
 // Complexidade: O(n log(n))
 
 bool ccw(pt &p, pt &a, pt &b, bool collinear = 0) {
     pt p1 = a - p;
     pt p2 = b - p;
     return collinear ? (p2 ^ p1) <= 0 : (p2 ^ p1) < 0;
 }
 
 void sort_by_angle(vector<pt>& v) { // sorta o vetor por angulo em relacao ao pivo
     pt p0 = *min_element(begin(v), end(v));
     sort(begin(v), end(v), [&](pt &l, pt &r) { // sorta clockwise
         pt p1 = l - p0;
         pt p2 = r - p0;
         ll c1 = p1 ^ p2;
         return c1 < 0 || ((c1 == 0) && p0.dist2(l) < p0.dist2(r));
     });
 }
 
 vector<pt> convex_hull(vector<pt> v, bool collinear = 0) {
     int n = size(v);
 
     sort_by_angle(v);
 
     if (collinear) {
         for (int i = n - 2; i >= 0; i--) { // reverte o ultimo lado do poligono
             if (ccw(v[0], v[n - 1], v[i])) {
                 reverse(begin(v) + i + 1, end(v));
                 break;
             }
         }
     }
 
     vector<pt> ch{v[0], v[1]};
     for (int i = 2; i < n; i++) {
         while (ch.size() > 2 && (ccw(ch.end()[-2], ch.end()[-1], v[i], !collinear))) ch.pop_back();
         ch.emplace_back(v[i]);
     }
 
     return ch;
 }
\end{lstlisting}

\pagebreak


%%%%%%%%%%%%%%%%%%%%
%
% Extra
%
%%%%%%%%%%%%%%%%%%%%

\section{Extra}

\subsection{Config do Vim}
\begin{lstlisting}
// .vimrc

 set nu
 set ai
 set ts=4
 set sw=4
 set so=10
 filet plugin indent on
 inoremap {} {}<Left><Return><Up><End><Return> 
 
 au BufReadPost * if line("'\"") > 0 && line("'\"") <= line("$") | exe "normal! g'\"" | endif
\end{lstlisting}

\subsection{Custom Hash}
\begin{lstlisting}
// Hash personalizado pra evitar colisao no unordered_map
// Uso: unordered_map<int, int, custom_hash> mapa;

 struct custom_hash {
     static uint64_t splitmix64(uint64_t x) {
         x += 0x9e3779b97f4a7c15;
         x = (x ^ (x >> 30)) * 0xbf58476d1ce4e5b9;
         x = (x ^ (x >> 27)) * 0x94d049bb133111eb;
         return x ^ (x >> 31);
     }
 
     size_t operator()(uint64_t x) const {
         static const uint64_t FIXED_RANDOM = chrono::steady_clock::now().time_since_epoch().count();
         return splitmix64(x + FIXED_RANDOM);
     }
 };
\end{lstlisting}

\subsection{Gerador aleatorio de casos}
\begin{lstlisting}
 #include <bits/stdc++.h>
 using namespace std;
 typedef long long ll;
 mt19937 rng(chrono::steady_clock::now().time_since_epoch().count());
 
 ll uniform(ll l, ll r) {
     uniform_int_distribution<int> uid(l, r);
     return uid(rng);
 }
 
 int main(){
     cout << uniform(1, 10) << endl;
 }
\end{lstlisting}

\subsection{Mint}
\begin{lstlisting}
// Inteiro automaticamente modulado

 template<int mod> struct Mint {
     int val;
     Mint(ll v = 0) { val = v % mod; if (val < 0) val += mod; }
     Mint pwr(Mint b, ll e) {
         Mint res;
         for (res = 1; e; e >>= 1, b = b * b) if (e & 1) res = res * b;
         return res;
     }
     bool operator==(Mint o) { return val == o.val; }
     bool operator<(Mint o) const { return val < o.val; }
     friend Mint operator*(Mint a, Mint o) { return (ll)a.val * o.val; }
     friend Mint operator+(Mint a, Mint o) {
         a.val += o.val;
         if (a.val >= mod) a.val -= mod;
         return a;
     }
     friend Mint operator-(Mint a, Mint o) {
         a.val -= o.val;
         if (a.val < 0) a.val += mod;
         return a;
     }
     friend Mint operator^(Mint a, ll o) { return a.pwr(a, o); }
     friend Mint operator/(Mint a, Mint o) { return a * (o ^ (mod - 2)); }
 };
 
 const int mod = 998244353;
 using mint = Mint<mod>;
\end{lstlisting}

\subsection{Rand C++}
\begin{lstlisting}
 mt19937 rng(chrono::steady_clock::now().time_since_epoch().count());
\end{lstlisting}

\subsection{Script de stress test}
\begin{lstlisting}
 set -e
 g++ -O2 code.cpp -o code
 g++ -O2 brute.cpp -o brute
 g++ -O2 gen.cpp -o gen
 
 for((i = 1; ; ++i)); do
     ./gen > in
     ./code < in > myout
     ./brute < in > out
     diff myout out > /dev/null || break
     echo "OK: "  $i
 done
 
 echo "WA:"
 cat in
 echo "Myout:"
 cat myout
 echo "Out:"
 cat out
\end{lstlisting}

\subsection{Script pra rodar C++}
\begin{lstlisting}
// chmod +x run
// ./run A.cpp

 #!/bin/bash
 g++ --std=c++20 -Wall -O2 -DNTJ -fsanitize=address,undefined $1 && ./a.out
\end{lstlisting}

\subsection{Template C++}
\begin{lstlisting}
 #include <bits/stdc++.h>
 #define endl '\n'
 
 using namespace std;
 typedef long long ll;
 
 void solve() {}
 
 signed main() {
     ios_base::sync_with_stdio(0);
     cin.tie(0);
     solve();
 }
\end{lstlisting}

\subsection{Template de debug simples}
\begin{lstlisting}
 void _print() {}
 template <typename T, typename... U> void _print(T a, U... b) {
     if (sizeof...(b)) {
         cerr << a << ", ";
         _print(b...);
     } else
         cerr << a;
 }
 #ifdef NTJ
 #define debug(x...) cerr << "[" << #x << "] = [", _print(x), cerr << "]" << endl
 #else
 #define debug(...)
 #endif
\end{lstlisting}

\end{document}
